\documentclass{ees}

\shortname{\rlap{\hspace{.85em}°}Tuma}
\shorttitle{Lamentationes}

\begin{document}

\eesTitlePage

\def\eesCommentaryAfterToe{
  I
  Incipit lamentatio Ieremiae Prophetae.
  Aleph. Quomodo sedet sola civitas plena populo. Facta est quasi vidua domina gentium. Princeps provinciarum facta est sub tributo.
  Beth. Plorans ploravit in nocte, et lacrymae eius in maxillis eius. Non est qui consoletur eam ex omnibus charis eius. Omnes amici eius spreverunt eam et facti sunt ei inimici.
  Chimel. Migravit Iudas propter afflictionem et multitudinem servitutis. Habitavit inter gentes nec invenit requiem. Omnes persecutores eius apprehenderunt eam inter angustias.
  Ierusalem, Ierusalem, convertere ad Dominum Deum tuum.
  (\bibleverse{Lam}(1:1-3))

  II
  Vau. Et egressus est a filia Sion omnis decor eius. Facti sunt principes eius velut arietes non invenientes pascua. Et abierunt absque fortitudine ante faciem subsequentis.
  Zain. Recordata est Ierusalem dierum afflictionis suae et praevaricationis, omnium desiderabilium suorum, quae habuerat a diebus antiquis, cum caderet populus eius in manu hostili, et non esset auxiliator. Viderunt eam hostes et deriserunt Sabbata eius.
  Ierusalem, Ierusalem, convertere ad Dominum Deum tuum.
  (\bibleverse{Lam}(1:6-7))

  III
  Iod. Manum suam misit hostis ad omnia desiderabilia eius. Quia vidit gentes ingressas sanctuarium suum, de quibus praeceperas, ne intrarent in ecclesiam tuam.
  Caph. Omnis populus eius gemens et quaerens panem. Dederunt pretiosa quaeque pro cibo ad refocillandam animam. Vide, Domine, et considera, quoniam facta sum vilis!
  [Lamed.] O vos omnes, qui transitis per viam, attendite et videte, si est dolor sicut dolor meus.
  Ierusalem, Ierusalem, convertere ad Dominum Deum tuum.
  (\bibleverse{Lam}(1:10-12))
}

\eesCriticalReport{
  – & – & org & Organ part and bass figures have been added by the editor. \\
  1 & 51 & S 2 & last \quarterNote\ in \B1: f′4 \\
    & 73–78 & T & in \B1 written one second higher \\
  2 & 28 & B & 2nd \wholeNote\ in \B1: \flat B1 \\
    & 91 & A & 4th \halfNote\ in \B1: f′2 \\
    & 100 & A & 2nd \wholeNote\ in \B1: c′1 \\
    & 101 & A & 2nd \wholeNote\ in \B1: e′1 \\
    & 109f & A & 2nd \halfNote\ of bar 109 in \B1: a2–\flat b2;
                 1st \wholeNote\ of bar 110: c′1 \\
    & 145 & S & 4th \halfNote\ in \B1: \flat b′2 \\
  3 & 84 & T & 1st \wholeNote\ in \B1: \flat b1 \\
    & 88f & T & bars in \B1: d′1–e′1 and f′1–c′1
}

\eesToc{}

\eesScore

\end{document}


In -- ci -- pit la -- men -- ta -- ti -- o Ie -- re -- mi -- ae Pro -- phe -- tae.
A -- leph. Quo -- mo -- do se -- det so -- la ci -- vi -- tas ple -- na po -- pu -- lo. Fa -- cta est qua -- si vi -- du -- a do -- mi -- na gen -- ti -- um. Prin -- ceps pro -- vin -- ci -- a -- rum fa -- cta est sub tri -- bu -- to.
Beth. Plo -- rans plo -- ra -- vit in no -- cte, et la -- cry -- mae e -- ius in ma -- xil -- lis e -- ius. Non est qui con -- so -- le -- tur e -- am ex o -- mni -- bus cha -- ris e -- ius. O -- mnes a -- mi -- ci e -- ius spre -- ve -- runt e -- am et fa -- cti sunt e -- i in -- i -- mi -- ci.
Chi -- mel. Mi -- gra -- vit Iu -- das pro -- pter af -- fli -- cti -- o -- nem et mul -- ti -- tu -- di -- nem ser -- vi -- tu -- tis. Ha -- bi -- ta -- vit in -- ter gen -- tes nec in -- ve -- nit re -- qui -- em. O -- mnes per -- se -- cu -- to -- res e -- ius ap -- pre -- hen -- de -- runt e -- am in -- ter an -- gu -- sti -- as.
Ie -- ru -- sa -- lem, Ie -- ru -- sa -- lem, con -- ver -- te -- re ad Do -- mi -- num De -- um tu -- um.

Vau. Et e -- gres -- sus est a fi -- li -- a Si -- on o -- mnis de -- cor e -- ius. Fa -- cti sunt prin -- ci -- pes e -- ius ve -- lut a -- ri -- e -- tes non in -- ve -- ni -- en -- tes pa -- scu -- a. Et ab -- i -- e -- runt abs -- que for -- ti -- tu -- di -- ne an -- te fa -- ci -- em sub -- se -- quen -- tis.
Zain. Re -- cor -- da -- ta est Ie -- ru -- sa -- lem di -- e -- rum af -- fli -- cti -- o -- nis su -- ae et prae -- va -- ri -- ca -- ti -- o -- nis, o -- mni -- um de -- si -- de -- ra -- bi -- li -- um su -- o -- rum, quae ha -- bu -- e -- rat a di -- e -- bus an -- ti -- quis, cum ca -- de -- ret po -- pu -- lus e -- ius in ma -- nu ho -- sti -- li, et non es -- set au -- xi -- li -- a -- tor. Vi -- de -- runt e -- am ho -- stes et de -- ri -- se -- runt Sab -- ba -- ta e -- ius.
Ie -- ru -- sa -- lem, Ie -- ru -- sa -- lem, con -- ver -- te -- re ad Do -- mi -- num De -- um tu -- um.

Iod. Ma -- num su -- am mi -- sit ho -- stis ad o -- mni -- a de -- si -- de -- ra -- bi -- li -- a e -- ius. Qui -- a vi -- dit gen -- tes in -- gres -- sas san -- ctu -- a -- ri -- um su -- um, de qui -- bus prae -- ce -- pe -- ras, ne in -- tra -- rent in ec -- cle -- si -- am tu -- am.
Caph. O -- mnis po -- pu -- lus e -- ius ge -- mens et quae -- rens pa -- nem. De -- de -- runt pre -- ti -- o -- sa quae -- que pro ci -- bo ad re -- fo -- cil -- lan -- dam a -- ni -- mam. Vi -- de, Do -- mi -- ne, et con -- si -- de -- ra, quo -- ni -- am fa -- cta sum vi -- lis!
[La -- med.] O vos o -- mnes, qui trans -- i -- tis per vi -- am, at -- ten -- di -- te et vi -- de -- te, si est do -- lor si -- cut do -- lor me -- us.
Ie -- ru -- sa -- lem, Ie -- ru -- sa -- lem, con -- ver -- te -- re ad Do -- mi -- num De -- um tu -- um.
